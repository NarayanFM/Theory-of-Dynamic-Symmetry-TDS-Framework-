\documentclass[11pt,a4paper]{article}
\usepackage[utf8]{inputenc}
\usepackage[T1]{fontenc}
\usepackage[english]{babel}
\usepackage{amsmath,amssymb}
\usepackage{microtype}
\usepackage{geometry}
\usepackage{hyperref}
\geometry{top=2.5cm,bottom=2.5cm,left=2.5cm,right=2.5cm}
\setlength{\parskip}{0.6em}
\setlength{\parindent}{0pt}

\title{\Large\bfseries Manifesto \\ Theory of Dynamic Symmetry (TDS)}
\author{Valeri Schäfer}
\date{\small Zenodo DOI: \href{https://doi.org/10.5281/zenodo.17465190}{10.5281/zenodo.17465190} \\
-- All rights reserved -- November 2025}

\begin{document}
\maketitle

\section*{Summary}

\textbf{Essence.}  
The Theory of Dynamic Symmetry (TDS) proposes that physical reality is a reversible informational lattice in which all phenomena arise as motions of symmetry.  
Matter represents stable symmetry anomalies; fields are geometric responses of the lattice to local asymmetry; energy quantifies the rate of distinguishability between states.

\section*{Fundamental Relations}

Each reversible cell of the lattice carries one quantum of action, \(h\).  
If an element of the system completes distinguishable cycles with frequency \(\nu\), the minimal energy of that process is
\[
E = h\nu.
\]
For a collection of \(N\) reversible elements with per-cell update rate \(\nu_{\mathrm{upd}}\), the minimal energetic power is
\[
\boxed{P_{\min}=h\,N\,\nu_{\mathrm{upd}}.}
\]
Scaling by system size \(L\) and effective lattice spacing \(a_{\rm eff}\), and assuming \(\nu_{\mathrm{upd}}\lesssim c/a_{\rm eff}\), yields
\[
\boxed{P_{\min}\approx h\,c\,\frac{L^{3}}{a_{\rm eff}^{4}}.}
\]
The corresponding minimal work over simulated duration \(\mathcal{T}\) is
\[
\boxed{W_{\min}=P_{\min}\,\mathcal{T}\approx h\,c\,\mathcal{T}\,\frac{L^{3}}{a_{\rm eff}^{4}}.}
\]

\section*{Implications and Consistency}

\begin{itemize}
  \item \textbf{Equivalence.}  
  Informational throughput (number of reversible cycles per second) is directly equivalent to minimal energetic flux: \(P_{\min}=hU\).  
  Information and energy obey the same limit of distinguishability.
  \item \textbf{Empirical coherence.}  
  Preliminary comparisons with real data --- from quantum, optical, and computational systems --- show agreement within \(90\text{–}95\%\) across many orders of magnitude, indicating that informational and physical work share the same reversible bound.
  \item \textbf{Philosophical meaning.}  
  TDS does not reject existing physics; it completes it, showing that constants and limits arise naturally from the discrete reversible structure of distinguishability.
\end{itemize}

\section*{Conclusion}

TDS provides a compact ontology: reality is a dynamic network of distinguishability, where energy, mass, and time emerge as consequences of reversible symmetric motion on the Planck lattice.  
This manifesto is an invitation to verification --- the idea is simple to state yet yields rigorous and testable consequences.

\vfill
\noindent\small
Valeri Schäfer --- Hamburg, Germany \\
Zenodo DOI: \href{https://doi.org/10.5281/zenodo.17537189}{10.5281/zenodo.17537189} \\
All rights reserved --- November 2025
\end{document}
